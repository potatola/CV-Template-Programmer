%%%%%%%%%%%%%%%%%%%%%%%%%%%%%%%%%%%%%%%%%
% Medium Length Professional CV
% LaTeX Template
% Version 3.0 (2015-6-22)
%
% This template has been downloaded from:
% http://www.LaTeXTemplates.com
% Original author:
% Trey Hunner (http://www.treyhunner.com/)
% Modified by:
% Yufeng Geng
%
% Important note:
% This template requires the resume.cls file to be in the same directory as the
% .tex file. The resume.cls file provides the resume style used for structuring the
% document.
%
%%%%%%%%%%%%%%%%%%%%%%%%%%%%%%%%%%%%%%%%%

%----------------------------------------------------------------------------------------
%	PACKAGES AND OTHER DOCUMENT CONFIGURATIONS
%----------------------------------------------------------------------------------------

\documentclass{resume} % Use the custom resume.cls style

\usepackage{CJK}

\begin{CJK*}{GBK}{song}

\name{Yufeng Geng} % Your name
\address{Room 402, Building 4241, Peking University, Haidian District, Beijing 100871, P.R. China, 100871} % Your address
\address{gengyufeng0918@gmail.com ~ (+86)131-XXXX-XXXX} % Your phone number and email

\begin{document}

%----------------------------------------------------------------------------------------
%	EDUCATION SECTION
%----------------------------------------------------------------------------------------

\begin{rSection}{Education}

{\bf Peking University}, School of Electronics Engineering and Computer Science, Beijing \hfill {\em 2013.9-Present} \\
M.S. in Computer Applied Technology, Grade:\emph{86.32/100}

{\bf Peking University}, School of Electronics Engineering and Computer Science, Beijing \hfill {\em 2009.9-2013.7} \\
B.S. in Computer Science, GPA:\emph{3.46/4.0} \\
B.S. in Economy

\end{rSection}

%----------------------------------------------------------------------------------------
%	WORK EXPERIENCE SECTION
%----------------------------------------------------------------------------------------

\begin{rSection}{Projects}

\begin{rSubsection}{UEP for Video Streaming}{2014.10��2015.2}{Peking University}{}
\item Established the mathematics model of an FEC redundancy allocation approach. Transformed it as an optimization problem and resolved it with a suboptimal hill-climbing scheme.
\item Implemented a Matlab demo system which simulates the video codec (with JM18), our proposed FEC codec and network transmission with packet loss.
\item The PSNR performance was improved by more than 5 dB.
\item \emph{Paper}: Yufeng Geng, Xinggong Zhang, Chao Zhou, Zongming Guo, Unequall Error Protection For Real-time Video Streaming Using Expanding Window Reed-Solomon Code, Accepted by ICIP 2015.
\item \emph{Patent}: Redundancy allocation scheme for Reed-Solomon coding in real-time video streaming, 201510170234.5.
\end{rSubsection}

%------------------------------------------------

\begin{rSubsection}{Video Bitrate adaptation}{2014.3-2014.10}{Peking University}{}
\item Designed a novel queuing-delay based rate control algorithm, and employed a closed-loop cybernetic model for more stable and agile performance.
\item Implemented our algorithm as a new rate control module for Linphone (an C++ open-source VOIP software), which achieved 3dB or more gains in PSNR, and better performance on bandwidth utilization and flow stability.
\item \emph{Patent}: Video bitrate adaptation algorithm and system, 201510208870.2.
\end{rSubsection}

%------------------------------------------------

\begin{rSubsection}{Community365 Android Application}{2013.2-2013.8}{Peking University}{}
\item Worked with some other students to build an Android application for community social networking. This application allows users in the same community to serve each other with functions like sharing used items, sharing information, organizing activities.
\item Independently implemented UI of the main logic, including swipe views with tabs, event publishing and displaying with pictures, listviews with search function and so on.
\item Responsible for communication with our test users, promote our application, collect feedback and form software requirements.
\item Participated in the whole process of the application design and development, including logic, UI design, background interface and implementation.
\end{rSubsection}

\end{rSection}

%----------------------------------------------------------------------------------------
%	TECHNICAL STRENGTHS SECTION
%----------------------------------------------------------------------------------------

\begin{rSection}{Skills}

\begin{rList}
\item Solid knowledge in C++, Java.
\item Familiar with Python, Shell, Matlab, HTML, MySQL.
\item Solid knowledge in Software Engineering, Algorithms, Data Structures.
\item Familiar with computer networks, video streaming.
\item Fluent in oral and written English, CET-6 599/710
\end{rList}

\end{rSection}

%----------------------------------------------------------------------------------------
%	EXAMPLE SECTION
%----------------------------------------------------------------------------------------

\begin{rSection}{awards}

\begin{rList}
\item Tencent Innovative Scholarship \hfill {\em 2014}
\item Kwang-Hua Scholarship / Outstanding Graduate of Peking University(Top 10\% of PKU) \hfill {\em 2013}
\item WuSi Scholarship of Peking University / Founder Scholarship \hfill {\em Earlier}
\end{rList}

\end{rSection}

%----------------------------------------------------------------------------------------
\end{CJK*}
\end{document}
